\documentclass{article}
\usepackage[top=1in, right=1in, left=1in, bottom=1.5in]{geometry}

\usepackage{amsmath,amsthm,amsfonts,amssymb,amscd}
\usepackage{listings}
\usepackage{hyperref}
\usepackage{xcolor}
\usepackage{xr}

\usepackage{enumerate} 
\usepackage{physics}
\usepackage{fancyhdr}
\usepackage{hyperref}
\usepackage{graphicx}
\usepackage{tcolorbox}
\usepackage{catchfile}

% hyperref
\hypersetup{
  colorlinks=true,
  linkcolor=blue,
  linkbordercolor={0 0 1}
}

\usepackage{tcolorbox}
\newtcolorbox{contrib_box}{colback=red!5!white,colframe=red!75!black}
\newcommand{\contrib}{{\begin{contrib_box}This proof is in ``contrib'' and is still in the vetting process.\end{contrib_box}}} 

% asOfCommit macro to version a code dependency
\newcommand{\linkOpendpBlob}[3]{\href{https://github.com/opendp/opendp/blob/#1/#2#3}{#3 at commit #1}}
\immediate\write18{git rev-parse --show-prefix > out/cwd.txt}
\CatchFileDef\GitWorkingDir{cwd.txt}{\endlinechar=-1}
\makeatletter
\newcommand{\asOfCommit}[2]{%
    \linkOpendpBlob{#2}{\GitWorkingDir}{#1}
    \setbox0=\hbox{\input|"grep '^#2.*' <<< $(git --no-pager log -n1 --pretty='\@percentchar H' #1)"\unskip}\ifdim\wd0=0pt
      (outdated\footnote{See new changes with \texttt{git diff #2..\input|"git --no-pager log -n1 --pretty='\@percentchar h' #1}\texttt{\GitWorkingDir#1}})\else
      (up to date)\fi
}

\newcommand{\vettingPR}[1]{\href{https://github.com/opendp/opendp/pull/#1}{Pull Request \##1}}

% minted (pseudocode)
\definecolor{codegreen}{rgb}{0,0.6,0}
\definecolor{codegray}{rgb}{0.5,0.5,0.5}
\definecolor{codepurple}{rgb}{0.58,0,0.82}
\definecolor{backcolour}{rgb}{0.95,0.95,0.92}

\lstdefinestyle{mystyle}{
    backgroundcolor=\color{backcolour},   
    commentstyle=\color{codegreen},
    keywordstyle=\color{magenta},
    numberstyle=\tiny\color{codegray},
    stringstyle=\color{codepurple},
    basicstyle=\ttfamily\footnotesize,
    breakatwhitespace=false,         
    breaklines=true,                 
    captionpos=b,                    
    keepspaces=true,                 
    numbers=left,                    
    numbersep=5pt,                  
    showspaces=false,                
    showstringspaces=false,
    showtabs=false,                  
    tabsize=2
}

\lstset{style=mystyle}


% common commands
\theoremstyle{definition}
\newtheorem{theorem}{Theorem}[section]
\newtheorem{lemma}[theorem]{Lemma}
\newtheorem{definition}[theorem]{Definition}
\newtheorem{warning}{Warning}
\newtheorem{corollary}{Corollary}
\newtheorem{proposition}{Proposition}
\newtheorem{remark}{Remark}
\newtheorem{observation}{Observation}
\newtheorem{note}{Note}

\newcommand{\vicki}[1]{{ {\color{olive}{(vicki)~#1}}}}
\newcommand{\hanwen}[1]{{ {\color{purple}{(hanwen)~#1}}}}
\newcommand{\zach}[1]{{ {\color{red}{(zach)~#1}}}}

\newcommand{\MultiSet}{\mathrm{MultiSet}}
\newcommand{\len}{\mathrm{len}}
\newcommand{\din}{\texttt{d\_in}}
\newcommand{\dout}{\texttt{d\_out}}
\newcommand{\T}{\texttt{T} }
\newcommand{\F}{\texttt{F} }
\newcommand{\Relation}{\texttt{Relation}}
\newcommand{\X}{\mathcal{X}}
\newcommand{\Y}{\mathcal{Y}}
\newcommand{\True}{\texttt{True}}
\newcommand{\False}{\texttt{False}}
\newcommand{\clamp}{\texttt{clamp}}
\newcommand{\function}{\texttt{function}}
\newcommand{\float}{\texttt{float }}
\newcommand{\questionc}[1]{\textcolor{red}{\textbf{Question:} #1}}


\title{\texttt{mod domains}}
\author{S\'ilvia Casacuberta, Grace Tian, Connor Wagaman}
\date{}

\begin{document}

\maketitle

\contrib

\section{Vetting history}
\begin{itemize}
    \item \vettingPR{491}
\end{itemize}

\section{Definitions}
A \emph{data domain} is a representation of the set of values on which the function associated with a transformation or measurement can operate.  Each metric (see section \ref{sec:metrics}) is associated with certain data domains. Types used for implementing domains in OpenDP have trait \texttt{Domain} (defined in definition \ref{defn:traits-domain}).
\begin{definition}[\texttt{AllDomain}]
\texttt{AllDomain(T)} is the domain of all values of type $\T$. \domainType{AllDomain[T]}
\end{definition}
For example, \texttt{AllDomain(u32)} is the domain of all values of type \texttt{u32}.
\begin{definition}[\texttt{IntervalDomain}] For any type \texttt{T} with trait \texttt{TotalOrd} (see definition \ref{defn:totalord}),\footnote{As of June 28, the OpenDP library requires the weaker condition of partial ordering (implements \texttt{PartialOrd}) instead.} \texttt{IntervalDomain(L:T, U:T)} is the domain of all values \texttt{v} of type \texttt{T} such that \texttt{L <= v} and \texttt{v <= U}, for a type \texttt{T} that has a total ordering (\texttt{T} has trait \texttt{TotalOrd}) and for values \texttt{L <= U} of type \texttt{T}. \domainType{IntervalDomain[T]}
\end{definition}
An important remark is that the Rust implementation of \texttt{IntervalDomain} checks that \texttt{L <= U}, and returns an error if \texttt{L > U}. Therefore, any transformation or measurement that uses \texttt{IntervalDomain} does not need to re-check this constraint and raise a possible exception for it. 
Note that, because both \texttt{L} and \texttt{U} are of type \texttt{T}, there is no need to explicitly pass \texttt{T}; the type \texttt{T} can be inferred. \texttt{IntervalDomain} is defined on any type that implements the trait \texttt{TotalOrd}. For example, \texttt{IntervalDomain(1:u32, 17:u32)} corresponds to a domain that contains all the \texttt{u32} values \texttt{v} such that \texttt{1 <= v} and \texttt{v <= 17}; it has type \texttt{IntervalDomain[u32]}.
\begin{definition}[\texttt{InherentNullDomain}] 
    \texttt{InherentNullDomain(inner\_domain:D)} is the domain of all values of data domain \texttt{inner\_domain} unioned with a \texttt{null} value. Note that this means that a domain may have only \emph{one} \texttt{null} value; as a result, applying \texttt{InherentNullDomain} to a domain already containing a \texttt{null} value will not affect the domain. \domainType{InherentNullDomain[D]}
\end{definition}
\begin{definition}[\texttt{SizedDomain}]
    \texttt{SizedDomain(inner\_domain:D, n:usize)} is the domain of all elements from domain \texttt{D} restricted to length \texttt{n}. \domainType{SizedDomain[D]}
\end{definition}
For example, \texttt{SizedDomain(VectorDomain(AllDomain(u32)), n)} is the domain of all vectors of length $\texttt{n}$ with elements of type \texttt{u32}.
\begin{definition}[\texttt{VectorDomain}]
\texttt{VectorDomain(inner\_domain:D)} is the domain of all vectors of elements drawn from domain \texttt{inner\_domain}. 
\domainType{VectorDomain[D]}
\end{definition}

% TODO: add to issue
% \subsection{Subdomains}
% Preliminary theorems:
% \begin{theorem}[Domain-metric compatibility inheritance.]
%     Given a domain \texttt{D}, for any subdomain \texttt{S} $\subseteq$ \texttt{D}, if \texttt{D} is compatible with metric \texttt{M} then \texttt{S} is compatible with metric \texttt{M}.
% \end{theorem}

\end{document}
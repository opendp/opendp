\documentclass{article}
\usepackage[top=1in, right=1in, left=1in, bottom=1.5in]{geometry}

\usepackage{amsmath,amsthm,amsfonts,amssymb,amscd}
\usepackage{listings}
\usepackage{hyperref}
\usepackage{xcolor}
\usepackage{xr}

\usepackage{enumerate} 
\usepackage{physics}
\usepackage{fancyhdr}
\usepackage{hyperref}
\usepackage{graphicx}
\usepackage{tcolorbox}
\usepackage{catchfile}

% hyperref
\hypersetup{
  colorlinks=true,
  linkcolor=blue,
  linkbordercolor={0 0 1}
}

\usepackage{tcolorbox}
\newtcolorbox{contrib_box}{colback=red!5!white,colframe=red!75!black}
\newcommand{\contrib}{{\begin{contrib_box}This proof is in ``contrib'' and is still in the vetting process.\end{contrib_box}}} 

% asOfCommit macro to version a code dependency
\newcommand{\linkOpendpBlob}[3]{\href{https://github.com/opendp/opendp/blob/#1/#2#3}{#3 at commit #1}}
\immediate\write18{git rev-parse --show-prefix > out/cwd.txt}
\CatchFileDef\GitWorkingDir{cwd.txt}{\endlinechar=-1}
\makeatletter
\newcommand{\asOfCommit}[2]{%
    \linkOpendpBlob{#2}{\GitWorkingDir}{#1}
    \setbox0=\hbox{\input|"grep '^#2.*' <<< $(git --no-pager log -n1 --pretty='\@percentchar H' #1)"\unskip}\ifdim\wd0=0pt
      (outdated\footnote{See new changes with \texttt{git diff #2..\input|"git --no-pager log -n1 --pretty='\@percentchar h' #1}\texttt{\GitWorkingDir#1}})\else
      (up to date)\fi
}

\newcommand{\vettingPR}[1]{\href{https://github.com/opendp/opendp/pull/#1}{Pull Request \##1}}

% minted (pseudocode)
\definecolor{codegreen}{rgb}{0,0.6,0}
\definecolor{codegray}{rgb}{0.5,0.5,0.5}
\definecolor{codepurple}{rgb}{0.58,0,0.82}
\definecolor{backcolour}{rgb}{0.95,0.95,0.92}

\lstdefinestyle{mystyle}{
    backgroundcolor=\color{backcolour},   
    commentstyle=\color{codegreen},
    keywordstyle=\color{magenta},
    numberstyle=\tiny\color{codegray},
    stringstyle=\color{codepurple},
    basicstyle=\ttfamily\footnotesize,
    breakatwhitespace=false,         
    breaklines=true,                 
    captionpos=b,                    
    keepspaces=true,                 
    numbers=left,                    
    numbersep=5pt,                  
    showspaces=false,                
    showstringspaces=false,
    showtabs=false,                  
    tabsize=2
}

\lstset{style=mystyle}


% common commands
\theoremstyle{definition}
\newtheorem{theorem}{Theorem}[section]
\newtheorem{lemma}[theorem]{Lemma}
\newtheorem{definition}[theorem]{Definition}
\newtheorem{warning}{Warning}
\newtheorem{corollary}{Corollary}
\newtheorem{proposition}{Proposition}
\newtheorem{remark}{Remark}
\newtheorem{observation}{Observation}
\newtheorem{note}{Note}

\newcommand{\vicki}[1]{{ {\color{olive}{(vicki)~#1}}}}
\newcommand{\hanwen}[1]{{ {\color{purple}{(hanwen)~#1}}}}
\newcommand{\zach}[1]{{ {\color{red}{(zach)~#1}}}}

\newcommand{\MultiSet}{\mathrm{MultiSet}}
\newcommand{\len}{\mathrm{len}}
\newcommand{\din}{\texttt{d\_in}}
\newcommand{\dout}{\texttt{d\_out}}
\newcommand{\T}{\texttt{T} }
\newcommand{\F}{\texttt{F} }
\newcommand{\Relation}{\texttt{Relation}}
\newcommand{\X}{\mathcal{X}}
\newcommand{\Y}{\mathcal{Y}}
\newcommand{\True}{\texttt{True}}
\newcommand{\False}{\texttt{False}}
\newcommand{\clamp}{\texttt{clamp}}
\newcommand{\function}{\texttt{function}}
\newcommand{\float}{\texttt{float }}
\newcommand{\questionc}[1]{\textcolor{red}{\textbf{Question:} #1}}

\title{\texttt{fn sample\_bernoulli\_exp}}
\author{Michael Shoemate}

\begin{document}
\maketitle

From \href{https://arxiv.org/pdf/2004.00010.pdf#subsection.5.1}{subsection 5.1} of \cite{CKS20}.

\section{Pseudocode}
\subsubsection*{Precondition}
$\texttt{x} \in \mathbb{Q} \land \texttt{x} > 0$

\subsubsection*{Implementation}        
\begin{lstlisting}[language=Python]
def sample_bernoulli_exp(x) -> bool:
    while x >= 1:
        if sample_bernoulli_exp1(1):
            x -= 1
        else: 
            return False
    return sample_bernoulli_exp1(x)
\end{lstlisting}

\subsubsection*{Postcondition}
\texttt{sample\_bernoulli\_exp} returns a boolean sample from $Bernoulli(exp(-x))$, or an error propagated from \texttt{sample\_bernoulli\_exp1}.

\section{Proof}

Let $r$ be the return value of \texttt{sample\_bernoulli\_exp(x)}, assuming the preconditions are met, and the function does not return an error.

\begin{theorem}
$r$ is a realization of $R \sim Bernoulli(exp(-x))$, where $x \in \mathbb{Q}$ and $x > 0$ \cite{CKS20}.
\end{theorem}

\begin{proof}
Let $B_i \sim Bernoulli(exp(-1))$ be the sequence of booleans returned from \texttt{sample\_bernoulli\_exp1(1)}, and $C \sim Bernoulli(exp(-(x - \lfloor x \rfloor)))$.
The function only returns $\top$ if $B_i = \top$ for all $i$ and $C = \top$.

\begin{align*}
    P[R = \top] &= P[B_1 = B_2 = ... = B_{\lfloor x \rfloor} = C = \top] \\
    &= \prod_{i=1}^{\lfloor x \rfloor} P[B_i = \top] P[C = \top] && \text{All $B_i$ are independent.} \\
    &= exp(-1)^{\lfloor x \rfloor} exp(\lfloor x \rfloor - x) \\
    &= exp(-x)
\end{align*}
\end{proof}

\bibliographystyle{alpha}
\bibliography{mod}

\end{document}
\documentclass{article}
\usepackage[top=1in, right=1in, left=1in, bottom=1.5in]{geometry}

\usepackage{amsmath,amsthm,amsfonts,amssymb,amscd}
\usepackage{listings}
\usepackage{hyperref}
\usepackage{xcolor}
\usepackage{xr}

\usepackage{enumerate} 
\usepackage{physics}
\usepackage{fancyhdr}
\usepackage{hyperref}
\usepackage{graphicx}
\usepackage{tcolorbox}
\usepackage{catchfile}

% hyperref
\hypersetup{
  colorlinks=true,
  linkcolor=blue,
  linkbordercolor={0 0 1}
}

\usepackage{tcolorbox}
\newtcolorbox{contrib_box}{colback=red!5!white,colframe=red!75!black}
\newcommand{\contrib}{{\begin{contrib_box}This proof is in ``contrib'' and is still in the vetting process.\end{contrib_box}}} 

% asOfCommit macro to version a code dependency
\newcommand{\linkOpendpBlob}[3]{\href{https://github.com/opendp/opendp/blob/#1/#2#3}{#3 at commit #1}}
\immediate\write18{git rev-parse --show-prefix > out/cwd.txt}
\CatchFileDef\GitWorkingDir{cwd.txt}{\endlinechar=-1}
\makeatletter
\newcommand{\asOfCommit}[2]{%
    \linkOpendpBlob{#2}{\GitWorkingDir}{#1}
    \setbox0=\hbox{\input|"grep '^#2.*' <<< $(git --no-pager log -n1 --pretty='\@percentchar H' #1)"\unskip}\ifdim\wd0=0pt
      (outdated\footnote{See new changes with \texttt{git diff #2..\input|"git --no-pager log -n1 --pretty='\@percentchar h' #1}\texttt{\GitWorkingDir#1}})\else
      (up to date)\fi
}

\newcommand{\vettingPR}[1]{\href{https://github.com/opendp/opendp/pull/#1}{Pull Request \##1}}

% minted (pseudocode)
\definecolor{codegreen}{rgb}{0,0.6,0}
\definecolor{codegray}{rgb}{0.5,0.5,0.5}
\definecolor{codepurple}{rgb}{0.58,0,0.82}
\definecolor{backcolour}{rgb}{0.95,0.95,0.92}

\lstdefinestyle{mystyle}{
    backgroundcolor=\color{backcolour},   
    commentstyle=\color{codegreen},
    keywordstyle=\color{magenta},
    numberstyle=\tiny\color{codegray},
    stringstyle=\color{codepurple},
    basicstyle=\ttfamily\footnotesize,
    breakatwhitespace=false,         
    breaklines=true,                 
    captionpos=b,                    
    keepspaces=true,                 
    numbers=left,                    
    numbersep=5pt,                  
    showspaces=false,                
    showstringspaces=false,
    showtabs=false,                  
    tabsize=2
}

\lstset{style=mystyle}


% common commands
\theoremstyle{definition}
\newtheorem{theorem}{Theorem}[section]
\newtheorem{lemma}[theorem]{Lemma}
\newtheorem{definition}[theorem]{Definition}
\newtheorem{warning}{Warning}
\newtheorem{corollary}{Corollary}
\newtheorem{proposition}{Proposition}
\newtheorem{remark}{Remark}
\newtheorem{observation}{Observation}
\newtheorem{note}{Note}

\newcommand{\vicki}[1]{{ {\color{olive}{(vicki)~#1}}}}
\newcommand{\hanwen}[1]{{ {\color{purple}{(hanwen)~#1}}}}
\newcommand{\zach}[1]{{ {\color{red}{(zach)~#1}}}}

\newcommand{\MultiSet}{\mathrm{MultiSet}}
\newcommand{\len}{\mathrm{len}}
\newcommand{\din}{\texttt{d\_in}}
\newcommand{\dout}{\texttt{d\_out}}
\newcommand{\T}{\texttt{T} }
\newcommand{\F}{\texttt{F} }
\newcommand{\Relation}{\texttt{Relation}}
\newcommand{\X}{\mathcal{X}}
\newcommand{\Y}{\mathcal{Y}}
\newcommand{\True}{\texttt{True}}
\newcommand{\False}{\texttt{False}}
\newcommand{\clamp}{\texttt{clamp}}
\newcommand{\function}{\texttt{function}}
\newcommand{\float}{\texttt{float }}
\newcommand{\questionc}[1]{\textcolor{red}{\textbf{Question:} #1}}

\title{\texttt{fn sample\_discrete\_gaussian}}
\author{Michael Shoemate}

\begin{document}
\maketitle

\contrib
Proves soundness of \texttt{fn sample\_discrete\_gaussian} in \asOfCommit{mod.rs}{0be3ab3e6}.
This proof is an adaptation of \href{https://arxiv.org/pdf/2004.00010.pdf#subsection.5.3}{subsection 5.3} of \cite{CKS20}.

\section{Vetting history}
\begin{itemize}
    \item \vettingPR{519}
\end{itemize}

\section{Pseudocode}
\subsubsection*{Precondition}
$\texttt{scale} \in \mathbb{Q} \land \texttt{scale} \geq 0$

\subsubsection*{Implementation}        
\begin{lstlisting}[language=Python]
def sample_discrete_gaussian(scale) -> int:
    if scale == 0:
        return 0
    
    t = floor(scale) + 1
    sigma2 = scale**2
    
    while True:
        candidate = sample_discrete_laplace(t)
        x = abs(candidate) - sigma2 / t
        bias = x**2 / (2 * sigma2)
        if sample_bernoulli_exp(bias):
            return candidate
\end{lstlisting}

\subsubsection*{Postcondition}
\texttt{sample\_discrete\_gaussian} returns an integer sample from $\mathcal{N}_\mathbb{Z}(0, scale^2)$, or an error propagated from \texttt{sample\_discrete\_laplace} or \texttt{sample\_bernoulli\_exp}.

\section{Proof}

First, define the target distribution.

\begin{definition}
    (Discrete Gaussian). Let $\mu, \sigma \in \mathbb{R}$ with $\sigma > 0$. 
    The discrete gaussian distribution with location $\mu$ and scale $\sigma$ is denoted $\mathcal{N}_\mathbb{Z}(\mu, \sigma^2)$. 
    It is a probability distribution supported on the integers and defined by \cite{CKS20}
\begin{equation*}
    \forall x \in \mathbb{Z} \quad  P[X = x] = \frac{e^{-\frac{(x - \mu)^2}{2\sigma^2}}}{\sum_{y\in\mathbb{Z}}e^{-\frac{(y - \mu)^2}{2\sigma^2}}} \quad \text{where } X \sim \mathcal{N}_\mathbb{Z}(\mu, \sigma^2)
\end{equation*}
\end{definition}


We must show that the return value of \texttt{sample\_discrete\_gaussian(scale)}, conditioned on not returning an error, is a sample from $\mathcal{N}_\mathbb{Z}(0, scale^2)$.

Let $t = \lfloor \sigma \rfloor + 1$. 
Now fix any iteration of the loop. 

\begin{lemma}
If $y$ is a realization of $Y \sim \mathcal{L}_\mathbb{Z}(0, \sigma)$, and $c$ is a realization of $C \sim Bernoulli(exp(-(|y| - \sigma^2 / t)^2 / (2 \sigma^2)))$, then
$E[C] = \frac{1 - e^{-1/\sigma}}{1 + e^{-1/\sigma}}e^{-\frac{\sigma^2}{2t^2}} \sum_{y\in \mathbb{Z}} e^{-\frac{y^2}{2\sigma^2}}$\cite{CKS20}.
\end{lemma}

\begin{proof}
\begin{align*}
    E[C] &= E[E[C|Y]] \\
    &= E[e^{-\frac{(|Y| - \sigma^2/t)^2}{2\sigma^2}}] && \text{since } E[Bernoulli(p)] = p \\
    &= \frac{1 - e^{-1/\sigma}}{1 + e^{-1/\sigma}} \sum_{y\in \mathbb{Z}} e^{-\frac{(|y| - \sigma^2/t)^2}{2\sigma^2} - |y|/t} && \text{expectation over } Y \sim \mathcal{L}_\mathcal{Z}(0, \sigma) \\
    &= \frac{1 - e^{-1/\sigma}}{1 + e^{-1/\sigma}}e^{-\frac{\sigma^2}{2t^2}} \sum_{y\in \mathbb{Z}} e^{-\frac{y^2}{2\sigma^2}}
\end{align*}
\end{proof}

% We now show that conditioning $Y$ on the success of C gives the desired output distribution.
\begin{theorem}
\label{P_Yy_CT} If $y$ is a realization of $Y \sim \mathcal{L}_\mathbb{Z}(0, \sigma)$ and $c$ is a realization of $C \sim Bernoulli(exp(-(|y| - \sigma^2 / t)^2 / (2 \sigma^2)))$, then
$P[Y=y | C=\top] = \frac{e^{-\frac{y^2}{2\sigma^2}}}{\sum_{y' \in \mathbb{Z}} e^{-\frac{y'^2}{2\sigma^2}}}$. That is, $Y|_{C=\top} \sim \mathcal{N}_\mathbb{Z}(0, \sigma^2)$ \cite{CKS20}.
\end{theorem}

\begin{proof}

\begin{align*}
    P[Y=y | C=\top] &= \frac{P[C=\top|Y=y]P[Y=y]}{P[C=\top]} && \text{Bayes' Theorem} \\
    &= \frac{e^-\frac{(|y| - \sigma^2/t)^2}{2\sigma^2} \frac{1 - e^{-1/t}}{1 + e^{-1/t}} e^{-|y|/t}}{E[C]} && \text{by definition of } \mathcal{L}_\mathbb{Z}(0, \sigma) \\
    &= \frac{e^-\frac{(|y| - \sigma^2/t)^2}{2\sigma^2} e^{-|y|/t}}{e^{-(\sigma/t)^2/2} \sum_{y' \in \mathbb{Z}} e^{-\frac{y'^2}{2\sigma^2}}} \\
    &= \frac{e^{-\frac{y^2}{2\sigma^2}}}{\sum_{y' \in \mathbb{Z}} e^{-\frac{y'^2}{2\sigma^2}}}
\end{align*}
\end{proof}

Since the function's return value is a draw from $Y \sim \mathcal{L}_\mathbb{Z}(0, scale)$ conditioned on $C = \top$, then by \ref{P_Yy_CT}, the return is distributed $\mathcal{N}_\mathbb{Z}(0, scale^2)$.

\bibliographystyle{alpha}
\bibliography{mod}

\end{document}
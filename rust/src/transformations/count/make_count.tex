\documentclass{article}
\usepackage[top=1in, right=1in, left=1in, bottom=1.5in]{geometry}

\usepackage{amsmath,amsthm,amsfonts,amssymb,amscd}
\usepackage{listings}
\usepackage{hyperref}
\usepackage{xcolor}
\usepackage{xr}

\usepackage{enumerate} 
\usepackage{physics}
\usepackage{fancyhdr}
\usepackage{hyperref}
\usepackage{graphicx}
\usepackage{tcolorbox}
\usepackage{catchfile}

% hyperref
\hypersetup{
  colorlinks=true,
  linkcolor=blue,
  linkbordercolor={0 0 1}
}

\usepackage{tcolorbox}
\newtcolorbox{contrib_box}{colback=red!5!white,colframe=red!75!black}
\newcommand{\contrib}{{\begin{contrib_box}This proof is in ``contrib'' and is still in the vetting process.\end{contrib_box}}} 

% asOfCommit macro to version a code dependency
\newcommand{\linkOpendpBlob}[3]{\href{https://github.com/opendp/opendp/blob/#1/#2#3}{#3 at commit #1}}
\immediate\write18{git rev-parse --show-prefix > out/cwd.txt}
\CatchFileDef\GitWorkingDir{cwd.txt}{\endlinechar=-1}
\makeatletter
\newcommand{\asOfCommit}[2]{%
    \linkOpendpBlob{#2}{\GitWorkingDir}{#1}
    \setbox0=\hbox{\input|"grep '^#2.*' <<< $(git --no-pager log -n1 --pretty='\@percentchar H' #1)"\unskip}\ifdim\wd0=0pt
      (outdated\footnote{See new changes with \texttt{git diff #2..\input|"git --no-pager log -n1 --pretty='\@percentchar h' #1}\texttt{\GitWorkingDir#1}})\else
      (up to date)\fi
}

\newcommand{\vettingPR}[1]{\href{https://github.com/opendp/opendp/pull/#1}{Pull Request \##1}}

% minted (pseudocode)
\definecolor{codegreen}{rgb}{0,0.6,0}
\definecolor{codegray}{rgb}{0.5,0.5,0.5}
\definecolor{codepurple}{rgb}{0.58,0,0.82}
\definecolor{backcolour}{rgb}{0.95,0.95,0.92}

\lstdefinestyle{mystyle}{
    backgroundcolor=\color{backcolour},   
    commentstyle=\color{codegreen},
    keywordstyle=\color{magenta},
    numberstyle=\tiny\color{codegray},
    stringstyle=\color{codepurple},
    basicstyle=\ttfamily\footnotesize,
    breakatwhitespace=false,         
    breaklines=true,                 
    captionpos=b,                    
    keepspaces=true,                 
    numbers=left,                    
    numbersep=5pt,                  
    showspaces=false,                
    showstringspaces=false,
    showtabs=false,                  
    tabsize=2
}

\lstset{style=mystyle}


% common commands
\theoremstyle{definition}
\newtheorem{theorem}{Theorem}[section]
\newtheorem{lemma}[theorem]{Lemma}
\newtheorem{definition}[theorem]{Definition}
\newtheorem{warning}{Warning}
\newtheorem{corollary}{Corollary}
\newtheorem{proposition}{Proposition}
\newtheorem{remark}{Remark}
\newtheorem{observation}{Observation}
\newtheorem{note}{Note}

\newcommand{\vicki}[1]{{ {\color{olive}{(vicki)~#1}}}}
\newcommand{\hanwen}[1]{{ {\color{purple}{(hanwen)~#1}}}}
\newcommand{\zach}[1]{{ {\color{red}{(zach)~#1}}}}

\newcommand{\MultiSet}{\mathrm{MultiSet}}
\newcommand{\len}{\mathrm{len}}
\newcommand{\din}{\texttt{d\_in}}
\newcommand{\dout}{\texttt{d\_out}}
\newcommand{\T}{\texttt{T} }
\newcommand{\F}{\texttt{F} }
\newcommand{\Relation}{\texttt{Relation}}
\newcommand{\X}{\mathcal{X}}
\newcommand{\Y}{\mathcal{Y}}
\newcommand{\True}{\texttt{True}}
\newcommand{\False}{\texttt{False}}
\newcommand{\clamp}{\texttt{clamp}}
\newcommand{\function}{\texttt{function}}
\newcommand{\float}{\texttt{float }}
\newcommand{\questionc}[1]{\textcolor{red}{\textbf{Question:} #1}}

\title{\texttt{fn make\_count}}
\author{S\'ilvia Casacuberta, Grace Tian, Connor Wagaman -- wagaman@college.harvard.edu}
\date{}

\begin{document}

\maketitle

\contrib
Proves soundness of \texttt{fn make\_count} in \asOfCommit{mod.rs}{f5bb719}.

\section{Vetting history}
\begin{itemize}
    \item \vettingPR{513}
\end{itemize}

\subsection{Versions of definitions documents}
\label{sec:versioned-docs}

When looking for definitions for terms that appear in this document, the following versions of the definitions documents should be used.

\begin{itemize}
    \item \textbf{Pseudocode definitions document:} This proof file uses the version of the pseudocode definitions document available as of September 23, 2021, which can be found at \href{https://github.com/opendp/whitepapers/blob/pseudocode-defns/pseudocode-defns/pseudocode_defns.pdf}{this link} (archived \href{https://web.archive.org/web/20210906201546/https://raw.githubusercontent.com/opendp/whitepapers/pseudocode-defns/pseudocode-defns/pseudocode_defns.pdf}{here}).
    
    \item \textbf{Proof definitions document:} This file uses the version of the proof definitions document available as of September 23, 2021, which can be found at \href{https://github.com/opendp/whitepapers/blob/proof-defns/proof-defns/proof_defns.pdf}{this link} (archived \href{https://web.archive.org/web/20210906201056/https://raw.githubusercontent.com/opendp/whitepapers/proof-defns/proof-defns/proof_defns.pdf}{here}). 
\end{itemize}

\section{Pseudocode}

\label{sec:pseudocode}

\subsection*{Preconditions}
\begin{itemize}

    \item \texttt{TIA} (atomic input type) is a type with trait \texttt{Primitive}. \texttt{Primitive} implies \texttt{TIA} has the trait bound:
    \begin{itemize}
        \item \texttt{CheckNull} so that \texttt{TIA} is a valid atomic type for \texttt{VectorDomain}
    \end{itemize}

    \item \texttt{TO} (output type) is a type with trait \texttt{Number}. \texttt{Number} further implies \texttt{TO} has the trait bounds:
    \begin{itemize}
        \item \texttt{CheckNull} so that \texttt{TO} is a valid atomic type for \texttt{AllDomain}
        \item \texttt{ExactIntCast<usize>} for casting a vector length index of type \texttt{usize} to \texttt{TO}. \texttt{ExactIntCast} further implies \texttt{TO} has the trait bound:
        \begin{itemize}
            \item \texttt{ExactIntBounds}, which gives the \texttt{MAX\_CONSECUTIVE} value of type \texttt{TO}
        \end{itemize}
        
        \item \texttt{One} provides a way to retrieve \texttt{TO}'s representation of 1
        \item \texttt{DistanceConstant<IntDistance>} to satisfy the preconditions of \texttt{new\_stability\_map\_from\_constant}
    \end{itemize}
\end{itemize}

\subsection*{Implementation}
\begin{lstlisting}[language = Python, escapechar=|]
def make_count():
    input_domain = VectorDomain(AllDomain(TIA))
    output_domain = AllDomain(TO) |\label{line:output-domain}|

    def function(data: Vec[TIA]) -> TO:|\label{line:TO-output}|
        try: |\label{line:try-catch}|
            return TO.exact_int_cast(len(data))
        except FailedCast:
            return TO.MAX_CONSECUTIVE |\label{line:except-return}|

    input_metric = SymmetricDistance()
    output_metric = AbsoluteDistance(TO)

    stability_map = new_stability_map_from_constant(TO.one()) |\label{line:stability-map}|

    return Transformation(
        input_domain, output_domain, function,
        input_metric, output_metric, stability_map)
\end{lstlisting}

\subsection*{Postcondition}
\texttt{make\_count} raises an exception or returns a valid \texttt{Transformation}.

\section{Proof}

\transformationTheorem{\texttt{(TIA, TO)}}{\texttt{make\_count}}

\begin{proof} \textbf{(Part 1 -- appropriate output domain).}
    The \texttt{output\_domain} is \texttt{AllDomain(TO)}, so it is sufficient to show that \texttt{function} always returns non-null values of type \texttt{TO}.
    By the definition of the \texttt{ExactIntCast} trait, \texttt{TO.exact\_int\_cast} always returns a non-null value of type \texttt{TO} or raises an exception.
    If an exception is raised, the function returns \texttt{TO.MAXIMUM\_CONSECUTIVE}, which is also a non-null value of type \texttt{TO}.
    Thus, in all cases, the function (from line \ref{line:try-catch}) returns a non-null value of type \texttt{TO}.
\end{proof}

\begin{proof} \textbf{(Part 2 -- domain-metric compatibility).}
    Our \texttt{input\_metric} of \texttt{SymmetricDistance} is compatible with any domain of the form \texttt{VectorDomain(inner\_domain)}, 
    and our \texttt{input\_domain} of \\\texttt{VectorDomain(AllDomain(TIA))} is of this form. 
    Therefore our \texttt{input\_domain} and \texttt{input\_metric} are compatible.

    Our \texttt{output\_metric} of \texttt{AbsoluteDistance(TO)} is compatible with any domain of the form \texttt{AllDomain(T)} where \texttt{T} has the trait \texttt{InfSub}, 
    and our \texttt{output\_domain} of \texttt{AllDomain(TO)} is of this form and \texttt{TO} has the necessary trait.
    Therefore our \texttt{input\_domain} and \texttt{input\_metric} are compatible.
\end{proof}

Before proceeding with proving the validity of the stability map, we provide a couple lemmas.

\begin{lemma}
    \label{dsym-sens}
    $|\function(u) - \function(v)| \leq |\texttt{len(u)} - \texttt{len(v)}|$.
\end{lemma}

\begin{proof}
    In line \ref{line:try-catch}, we know the argument to \texttt{TO.exact\_int\_cast} is non-negative and integral.
    Therefore, by the definition of \texttt{ExactIntCast}, the invocation of \texttt{TO.exact\_int\_cast} can only fail if the argument is greater than \texttt{TO.MAX\_CONSECUTIVE}.
    In this case, the value is replaced with \texttt{TO.MAX\_CONSECUTIVE}.
    Therefore, $\function(u) = min(\texttt{len(u)}, c)$, where $c = \texttt{TO.MAX\_CONSECUTIVE}$.
    We use this equality to prove the lemma:

    \begin{align*}
        |\function(u) - \function(v)| &= |min(\texttt{len(u)}, c) - min(\texttt{len(v)}, c)| \\
        &\leq |\texttt{len(u)} - \texttt{len(v)}| &&\text{since clamping is stable} \\
    \end{align*}
\end{proof}

\begin{lemma}
    \label{lemma:len-sum-equiv}
    For vector $v$ with each element $\ell\in v$ drawn from domain $\mathcal{X}$, $\texttt{len(v)} = \sum_{z\in\mathcal{X}} h_v(z)$.
\end{lemma}

\begin{proof}
    Every element $\ell \in v$ is drawn from domain $\mathcal{X}$, so summing over all $z\in \mathcal{X}$ will sum over every element $\ell\in x$. 
    Recall that the definition of \texttt{SymmetricDistance} states that $h_v(z)$ will return the number of occurrences of value $z$ in vector $v$.
    Therefore, $\sum_{z\in\mathcal{X}} h_v(z)$ is the sum of the number of occurrences of each unique value; 
    this is equivalent to the total number of items in the vector. 
    By the definition of \texttt{len} available in the pseudocode definitions document linked in section \ref{sec:versioned-docs}, 
    then, $\sum_{z\in\mathcal{X}} h_v(z)$ is equivalent to \texttt{len(v)}.
\end{proof}

\begin{proof} \textbf{(Part 3 -- stability map).} 
    Take any two elements $u, v$ in the \texttt{input\_domain} and any pair $(\din, \dout)$, 
    where \din\ has the associated type for \texttt{input\_metric} and \dout\ has the associated type for \texttt{output\_metric}.
    Assume $u, v$ are \din-close under \texttt{input\_metric} and that $\texttt{stability\_map}(\din) \leq \dout$. 
    These assumptions are used to establish the following inequality:
    \begin{align*}
        |\function(u) - \function(v)| &\leq |\texttt{len(u)} - \texttt{len(v)}| &&\text{by }\ref{dsym-sens} \\
        &= |\sum_{z\in \mathcal{X}} h_{\texttt{u}}(z) - \sum_{z\in \mathcal{X}} h_{\texttt{v}}(z)| &&\text{by } \ref{lemma:len-sum-equiv} \\
        &= |\sum_{z\in \mathcal{X}}\left(h_{\texttt{u}}(z) - h_{\texttt{v}}(z)\right)| &&\text{by algebra} \\
        &\leq \sum_{z\in \mathcal{X}}|h_{\texttt{u}}(z) - h_{\texttt{v}}(z)| &&\text{by triangle inequality} \\
        &= d_{Sym}(u, v) &&\text{by } \texttt{SymmetricDistance}\\
        &\leq \din &&\text{by the first assumption} \\
        &\leq \texttt{TO.inf\_cast}(\din) &&\text{by } \texttt{InfCast} \\
        &\leq \texttt{TO.one().inf\_mul(TO.inf\_cast(\din))} &&\text{by } \texttt{InfMul} \\
        &=\texttt{stability\_map}(\din) &&\text{by pseudocode line } \ref{line:stability-map} \\
        &\leq \dout &&\text{by the second assumption}
    \end{align*}

    It is shown that \function(u), \function(v) are \dout-close under \texttt{output\_metric}.
\end{proof}

\end{document}

\documentclass{article}
\usepackage[top=1in, right=1in, left=1in, bottom=1.5in]{geometry}

\usepackage{amsmath,amsthm,amsfonts,amssymb,amscd}
\usepackage{listings}
\usepackage{hyperref}
\usepackage{xcolor}
\usepackage{xr}

\usepackage{enumerate} 
\usepackage{physics}
\usepackage{fancyhdr}
\usepackage{hyperref}
\usepackage{graphicx}
\usepackage{tcolorbox}
\usepackage{catchfile}

% hyperref
\hypersetup{
  colorlinks=true,
  linkcolor=blue,
  linkbordercolor={0 0 1}
}

\usepackage{tcolorbox}
\newtcolorbox{contrib_box}{colback=red!5!white,colframe=red!75!black}
\newcommand{\contrib}{{\begin{contrib_box}This proof is in ``contrib'' and is still in the vetting process.\end{contrib_box}}} 

% asOfCommit macro to version a code dependency
\newcommand{\linkOpendpBlob}[3]{\href{https://github.com/opendp/opendp/blob/#1/#2#3}{#3 at commit #1}}
\immediate\write18{git rev-parse --show-prefix > out/cwd.txt}
\CatchFileDef\GitWorkingDir{cwd.txt}{\endlinechar=-1}
\makeatletter
\newcommand{\asOfCommit}[2]{%
    \linkOpendpBlob{#2}{\GitWorkingDir}{#1}
    \setbox0=\hbox{\input|"grep '^#2.*' <<< $(git --no-pager log -n1 --pretty='\@percentchar H' #1)"\unskip}\ifdim\wd0=0pt
      (outdated\footnote{See new changes with \texttt{git diff #2..\input|"git --no-pager log -n1 --pretty='\@percentchar h' #1}\texttt{\GitWorkingDir#1}})\else
      (up to date)\fi
}

\newcommand{\vettingPR}[1]{\href{https://github.com/opendp/opendp/pull/#1}{Pull Request \##1}}

% minted (pseudocode)
\definecolor{codegreen}{rgb}{0,0.6,0}
\definecolor{codegray}{rgb}{0.5,0.5,0.5}
\definecolor{codepurple}{rgb}{0.58,0,0.82}
\definecolor{backcolour}{rgb}{0.95,0.95,0.92}

\lstdefinestyle{mystyle}{
    backgroundcolor=\color{backcolour},   
    commentstyle=\color{codegreen},
    keywordstyle=\color{magenta},
    numberstyle=\tiny\color{codegray},
    stringstyle=\color{codepurple},
    basicstyle=\ttfamily\footnotesize,
    breakatwhitespace=false,         
    breaklines=true,                 
    captionpos=b,                    
    keepspaces=true,                 
    numbers=left,                    
    numbersep=5pt,                  
    showspaces=false,                
    showstringspaces=false,
    showtabs=false,                  
    tabsize=2
}

\lstset{style=mystyle}


% common commands
\theoremstyle{definition}
\newtheorem{theorem}{Theorem}[section]
\newtheorem{lemma}[theorem]{Lemma}
\newtheorem{definition}[theorem]{Definition}
\newtheorem{warning}{Warning}
\newtheorem{corollary}{Corollary}
\newtheorem{proposition}{Proposition}
\newtheorem{remark}{Remark}
\newtheorem{observation}{Observation}
\newtheorem{note}{Note}

\newcommand{\vicki}[1]{{ {\color{olive}{(vicki)~#1}}}}
\newcommand{\hanwen}[1]{{ {\color{purple}{(hanwen)~#1}}}}
\newcommand{\zach}[1]{{ {\color{red}{(zach)~#1}}}}

\newcommand{\MultiSet}{\mathrm{MultiSet}}
\newcommand{\len}{\mathrm{len}}
\newcommand{\din}{\texttt{d\_in}}
\newcommand{\dout}{\texttt{d\_out}}
\newcommand{\T}{\texttt{T} }
\newcommand{\F}{\texttt{F} }
\newcommand{\Relation}{\texttt{Relation}}
\newcommand{\X}{\mathcal{X}}
\newcommand{\Y}{\mathcal{Y}}
\newcommand{\True}{\texttt{True}}
\newcommand{\False}{\texttt{False}}
\newcommand{\clamp}{\texttt{clamp}}
\newcommand{\function}{\texttt{function}}
\newcommand{\float}{\texttt{float }}
\newcommand{\questionc}[1]{\textcolor{red}{\textbf{Question:} #1}}

\title{Old proof for the stability guarantee of \texttt{fn make\_clamp}}
\author{S\'ilvia Casacuberta}
\date{}

\begin{document}

\maketitle

\contrib

Salil suggested introducing the definition of row transform and adding a general lemma for its stability guarantee, as shown in Section 3.3 in \href{https://www.overleaf.com/project/60d214e390b337703d200982}{``List of definitions used in the proofs"}. However, we keep the longer old proof (case-by-case) for completeness, and in case any issues arise when revising the definition and theorems relating to row transforms

\begin{proof}
\smallskip
\textbf{(Stability guarantee).} Throughout the stability guarantee proof, we can assume that $\function(v)$ and $\function(w)$ are in the correct output domain, by the \textit{appropriate output domain property} shown above. 

Since by assumption $\Map(\din) \leq \dout$, by the \texttt{make\_clamp} stability map, we have that $\din \leq \dout$. 
Moreover, $v, w$ are assumed to be $\din$-close. By the definition of the symmetric difference metric, this is equivalent to stating that $d_{Sym}(v, w) = |\MultiSet(v) \Delta \MultiSet(w)| \leq \din$.

Let $\mathcal{X}$ be the domain of all elements of type \texttt{T}. By applying the histogram notation,\footnote{Note that there is a bijection between multisets and histograms, which is why the proof can be carried out with either notion. For further details, please consult \url{https://www.overleaf.com/project/60d214e390b337703d200982}.}  it follows that
\[
d_{Sym}(v, w) = \lVert h_{v} - h_{w}\rVert_1 = \sum_{z \in \mathcal{X}} |h_v(z) - h_w(z)| \leq \din \leq \dout.
\]

We now consider $\MultiSet(\function(v))$ and $\MultiSet(\function(w))$.
For each element $z \in \MultiSet(v) \cup \MultiSet(w)$, where $z$ has type \texttt{T}, if $z \in \MultiSet(v) \Delta \MultiSet(w)$, we will assume wlog that $z \in \MultiSet(v) \setminus \MultiSet(w)$. We consider the following cases:

\begin{enumerate}
    \item $z >$ \texttt{U} or $z <$ \texttt{L}: then, in the former case, $\clamp(z) =$ \texttt{U}. First consider the case when $z \in \MultiSet(v) \cup \MultiSet(w)$ with the same multiplicity in both multisets. Then, $|h_{\function(v)}(z) - h_{\function(w)}(z)| = 0$ because we have both $h_{\function(v)}(z) = 0$ and $h_{\function(w)}(z) = 0$. Thus the sum
    \[
    \sum_{z \in \mathcal{X}} |h_{\function(v)}(z) - h_{\function(w)}(z)|
    \]
    remains invariant, because the quantity $|h_{v}(z) - h_{w}(z)|$ is added to $|h_{\function(v)}(\texttt{U}) - h_{\function(w)}(\texttt{U})|$, given that $\clamp(z) = \texttt{U}$. 
    
    Suppose $z$ has multiplicity $k_v \geq 0$ in $\MultiSet(v)$ and multiplicity $k_w \geq 0$ in $\MultiSet(w)$, where $k_v \neq k_w$. After considering $z$, the value $h_{\function(v)}(\texttt{U})$ becomes $h_{\function(v)}(\texttt{U}) + k_v$, and $h_{\function(w)}(\texttt{U})$ becomes $h_{\function(w)}(\texttt{U}) + k_w$. Hence the quantity $|h_{\function(v)}(\texttt{U}) - h_{\function(w)}(\texttt{U})|$ increases by at most $|h_v(z) - h_w(z)|$, since, by the triangle inequality,
    \begin{align*}
         & |(h_{\function(v)}(\texttt{U}) + k_v) - (h_{\function(w)}(\texttt{U}) + k_w)| \leq
    \\[0.2cm]
         & \leq |h_{\function(v)}(\texttt{U}) - h_{\function(w)}(\texttt{U})| + |k_v - k_w| =
    \\[0.2cm]
        & = |h_{\function(v)}(\texttt{U}) - h_{\function(w)}(\texttt{U})| + |h_v(z) - h_w(z)|.
    \end{align*}
    The same argument applies whenever $z < \texttt{L}$.\footnote{The first subcase discussed here, i.e., when $k_v = k_w$, is also proven by the triangle inequality expression above, but it seemed clean to separate the case where the total sum remains invariant.}
    
    \item $z \in \texttt{(L, U)}$: then, $\clamp(z) = z$. Since $h_v(z) = h_{\function(v)}(z)$ and $h_v(w) = h_{\function(w)}(z)$, it follows that $|h_v(z) - h_w(z)| = |h_{\function(v)}(z) - h_{\function(w)}(z)|$. Hence the histogram count, i.e., the quantity
     \[
        \sum_{z \in \mathcal{X}} |h_{\function(v)}(z) - h_{\function(w)}(z)|,
    \]
    remains invariant.
    
    \item $z = \texttt{U}$ or $z = \texttt{L}$: in the former case, $\clamp(z) = \texttt{U}$. If $z \in \MultiSet(v) \cup \MultiSet(w)$ with the same multiplicity in both multisets, then the histogram count remains invariant under the addition of element $z$. Otherwise, if $z \in \MultiSet(v) \setminus \MultiSet(w)$, or if $z$ is in their union but with different multiplicity, then element $z$ can increase the quantity $|h_{\function(v)}(\texttt{U}) - h_{\function(w)}(\texttt{U})|$ by at most $|h_v(z)-h_w(z)|$, following the same reasoning with the triangle inequality as in case~2.
    
    The same argument applies whenever $z = \texttt{L}$.
\end{enumerate}

By aggregating the three cases above, we conclude that
\[
\sum_{z \in \mathcal{X}} |h_{\function(v)}(z) - h_{\function(w)}(z)| \leq \sum_{z \in \mathcal{X}} |h_v(z) - h_w(z)|.
\]
By the initial assumptions, we recall that $\din \leq \dout$, and that $v, w$ are $\din$-close. Then,
\[
\sum_{z \in \mathcal{X}} |h_{\function(v)}(z) - h_{\function(w)}(z)| \leq \sum_{z \in \mathcal{X}} |h_v(z) - h_w(z)| \leq \din \leq \dout.
\]
Therefore, 
\[
|\MultiSet(\function(v)) \Delta \MultiSet(\function(w))| \leq \dout,
\]
as we wanted to show.
\end{proof}

\end{document}
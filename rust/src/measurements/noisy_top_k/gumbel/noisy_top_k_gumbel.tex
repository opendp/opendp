\documentclass{article} 
% common styling and macros shared by all proof files

\usepackage[top=1in, right=1in, left=1in, bottom=1.5in]{geometry}

\usepackage{amsmath,amsthm,amsfonts,amssymb,amscd}
\usepackage{listings}
\usepackage{hyperref}
\usepackage{xcolor}
\usepackage{xr}

\usepackage{enumerate} 
\usepackage{physics}
\usepackage{fancyhdr}
\usepackage{hyperref}
\usepackage{graphicx}
\usepackage{tcolorbox}
\usepackage{catchfile}
\usepackage{pdftexcmds}
\usepackage[T1]{fontenc}

% hyperref
\hypersetup{
  colorlinks=true,
  linkcolor=blue,
  linkbordercolor={0 0 1}
}

% \contrib macro to indicate inclusion in "contrib".
\usepackage{tcolorbox}
\newtcolorbox{contrib_box}{colback=red!5!white,colframe=red!75!black}
\newcommand{\contrib}{{\begin{contrib_box}This proof resides in \textbf{``contrib''} because it has not completed the vetting process.\end{contrib_box}}} 

% asOfCommit macro to version a code dependency. Arguments:
%    #1: relative path to file you are dependent on
%    #2: commit hash it was last edited. If outdated, this should be the second hash in the footnoote. Otherwise,
%            git log -n 1 --pretty=format:%h -- path/to/file.rs
\makeatletter
\ifnum\pdf@shellescape=1
   % "private" command that builds a link to a blob
  \newcommand{\linkOpendpBlob}[3]{%
    \href{https://github.com/opendp/opendp/blob/#1/#2#3}{\path{#3} at commit #1}}

  % latex macro expansion has a separate phase for \input evaluation
  %     immediately evaluate a command to write a temp file to ./out containing the current directory
  \immediate\write18{[ ! -f out/cwd.txt ] && (mkdir -p out && git rev-parse --show-prefix | sed "s|_|\@backslashchar\@backslashchar\@backslashchar_|g" > out/cwd.txt)}
  %     ...and then retrieve the current working directory by loading the temp file
  \CatchFileDef\GitWorkingDir{out/cwd.txt}{\endlinechar=-1}

  % command for building the (up to date) or (outdated) status
  \newcommand{\fileStatus}[2]{%
  \setbox0=\hbox{\input|"git --no-pager log -n1 --pretty='\@percentchar H' #1 | grep -E '^#2.*'"\unskip}\ifdim\wd0=0pt
        (outdated\footnote{See new changes with \texttt{git diff #2..\input|"git --no-pager log -n1 --pretty='\@percentchar h' #1" \GitWorkingDir\path{#1}}})\else
        (up to date)\fi
  }

  \newcommand{\asOfCommit}[2]{%
      % permalink the target
      \linkOpendpBlob{#2}{\GitWorkingDir}{#1}
      % conditionally add (outdated) or (up to date) depending on matching commit hash
      \fileStatus{#1}{#2}%
  }
\else
  % simplified command if shell-escape not enabled
  \newcommand{\asOfCommit}[2]{#1 at commit #2 (unknown status\footnote{Shell-escape is not enabled. Enable \texttt{--shell-escape} to check if this proof is up-to-date with the code.})}
\fi
\makeatother

% \vettingPR macro to link a PR. Arguments:
%    #1: PR number
\newcommand{\vettingPR}[1]{\href{https://github.com/opendp/opendp/pull/#1}{Pull Request \##1}}

% for links to rustdoc items in OpenDP. Arguments:
%    #1: path to item, and designation as trait, struct, fn, etc.
%    #2: item name
\makeatletter
\ifnum\pdf@shellescape=1
  % latex macro expansion has a separate phase for \input evaluation
  %     immediately evaluate a command to write a temp file to ./out containing the base path
  \immediate\write18{[ ! -f out/rustdoc.txt ] && mkdir -p out && ([ -z `kpsewhich --var-value OPENDP_RUSTDOC_PORT` ] && echo "https://docs.rs/opendp/`head -n 1 \@backslashchar`git rev-parse --show-toplevel\@backslashchar`/VERSION | sed 's|0.0.0+development|latest|g'`" || echo "http://localhost:`kpsewhich --var-value OPENDP_RUSTDOC_PORT`") > out/rustdoc.txt}
  %     ...and then retrieve the base path by loading the temp file
  \CatchFileDef\OpenDPRustdocBase{out/rustdoc.txt}{\endlinechar=-1}
\else
  % if shell commands are not enabled, just claim latest
  \newcommand{\OpenDPRustdocBase}{https://docs.rs/opendp/latest}
\fi
\makeatother
\newcommand{\rustdoc}[2]{\href{\OpenDPRustdocBase/opendp/#1.#2.html}{\texttt{#2}}}

% for links to external dependencies. Arguments:
%    #1: crate name
%    #2: path to item, and designation as trait, struct, fn, etc.
%    #3: item name
\newcommand{\docsrs}[3]{\href{https://docs.rs/#1/latest/#1/#2.#3.html}{\texttt{#3}}}

% minted (pseudocode)
\definecolor{codegreen}{rgb}{0,0.6,0}
\definecolor{codegray}{rgb}{0.5,0.5,0.5}
\definecolor{codepurple}{rgb}{0.58,0,0.82}
\definecolor{backcolour}{rgb}{0.95,0.95,0.92}

\lstdefinestyle{mystyle}{
    backgroundcolor=\color{backcolour},   
    commentstyle=\color{codegreen},
    keywordstyle=\color{magenta},
    numberstyle=\tiny\color{codegray},
    stringstyle=\color{codepurple},
    basicstyle=\ttfamily\footnotesize,
    breakatwhitespace=false,         
    breaklines=true,                 
    captionpos=b,                    
    keepspaces=true,                 
    numbers=left,                    
    numbersep=5pt,                  
    showspaces=false,                
    showstringspaces=false,
    showtabs=false,                  
    tabsize=2
}

\lstset{style=mystyle}

% common commands
\theoremstyle{definition}
\newtheorem{theorem}{Theorem}[section]
\newtheorem{lemma}[theorem]{Lemma}
\newtheorem{definition}[theorem]{Definition}
\newtheorem{warning}{Warning}
\newtheorem{corollary}{Corollary}
\newtheorem{proposition}{Proposition}
\newtheorem{remark}{Remark}
\newtheorem{observation}{Observation}
\newtheorem{note}{Note}

\newcommand{\vicki}[1]{{ {\color{olive}{(vicki)~#1}}}}
\newcommand{\hanwen}[1]{{ {\color{purple}{(hanwen)~#1}}}}
\newcommand{\zach}[1]{{ {\color{red}{(zach)~#1}}}}

\newcommand{\MultiSet}{\mathrm{MultiSet}}
\newcommand{\len}{\mathrm{len}}
\newcommand{\din}{\texttt{d\_in}}
\newcommand{\dout}{\texttt{d\_out}}
\newcommand{\T}{\texttt{T} }
\newcommand{\F}{\texttt{F} }
\newcommand{\Map}{\texttt{Map}}
\newcommand{\X}{\mathcal{X}}
\newcommand{\Y}{\mathcal{Y}}
\newcommand{\True}{\texttt{True}}
\newcommand{\False}{\texttt{False}}
\newcommand{\clamp}{\texttt{clamp}}
\newcommand{\function}{\texttt{function}}
\newcommand{\float}{\texttt{float }}
\newcommand{\questionc}[1]{\textcolor{red}{\textbf{Question:} #1}}


\newcommand{\validTransformation}[2]{%
  For every setting of the input parameters #1 to #2 such that the given preconditions
  hold, #2 raises an exception (at compile time or run time) or returns a valid transformation. A valid transformation has the following properties:
  \begin{enumerate}
      \item \textup{(Appropriate output domain).} 
      For every element $v$ in \texttt{input\_domain}, $\function(v)$ is in \texttt{output\_domain} or raises a data-independent runtime exception.
      
      \item \textup{(Domain-metric compatibility).} 
      The domain \texttt{input\_domain} matches one of the possible domains listed in the definition of \texttt{input\_metric}, 
      and likewise \texttt{output\_domain} matches one of the possible domains listed in the definition of \texttt{output\_metric}.
      
      \item \textup{(Stability guarantee).} 
      For every pair of elements $u, v$ in \texttt{input\_domain} and for every pair $(\din, \dout)$, 
      where \din\ has the associated type for \texttt{input\_metric} and \dout\ has the associated type for \\ \texttt{output\_metric}, 
      if $u, v$ are \din-close under \texttt{input\_metric} and $\texttt{stability\_map}(\din) \leq \dout$, 
      then $\function(u), \function(v)$ are $\dout$-close under \texttt{output\_metric}.
  \end{enumerate}
}


\newcommand{\validMeasurement}[2]{%
  For every setting of the input parameters #1 to #2 such that the given preconditions
  hold, #2 raises an exception (at compile time or run time) or returns a valid measurement. A valid measurement has the following properties:
  \begin{enumerate}
      \item \textup{(Domain-metric compatibility).} 
      The domain \texttt{input\_domain} matches one of the possible domains listed in the definition of \texttt{input\_metric}.
      
      \item \textup{(Privacy guarantee).} 
      For every pair of elements $u, v$ in \texttt{input\_domain} and for every pair $(\din, \dout)$, 
      where \din\ has the associated type for \texttt{input\_metric} and \dout\ has the associated type for \\ \texttt{output\_measure}, 
      if $u, v$ are \din-close under \texttt{input\_metric} and $\texttt{privacy\_map}(\din) \leq \dout$, 
      then $\function(u), \function(v)$ are $\dout$-close under \texttt{output\_measure}.
  \end{enumerate}
}
 
\allowdisplaybreaks

\title{\texttt{fn report\_noisy\_top\_k}} 
\author{Michael Shoemate} 
\begin{document}  
\maketitle 
 
\section{Hoare Triple} 
\subsection*{Precondition} 
\subsubsection*{Compiler-verified}
Types consistent with pseudocode.

\begin{itemize}
    \item Generic \texttt{RV} implements \rustdoc{measurements/report_noisy_top_k/trait}{SelectionRV}.
\end{itemize}

\subsubsection*{Caller-verified}
\begin{enumerate}
    \item Elements of \texttt{x} are non-null.
    \item \texttt{scale} is positive.
\end{enumerate}
 
\subsection*{Pseudocode} 
\label{sec:python-pseudocode} 
\lstinputlisting[language=Python,firstline=2,escapechar=|]{./pseudocode/report_noisy_top_k.py} 
 
\subsection*{Postcondition} 

\begin{theorem}
    \label{postcondition}
    Returns a noninteractive function with no side-effects that, 
    when given a vector of non-null scores,
    returns the indices of the top k $z_i$,
    where each $z_i \sim RV(\mathrm{shift}=y_i, \mathrm{scale}=\texttt{scale})$,
    and each $y_i = -x_i$ if \texttt{optimize} is \texttt{min}, else $y_i = x_i$.

    If an error is returned, the error is data-dependent.
\end{theorem}

\begin{lemma}
    \label{lem:zero-scale}
    The postcondition holds when the scale is zero.
\end{lemma}

\begin{proof}[Proof of Lemma~\ref{lem:zero-scale}]
    The comparator on line \ref{fn-optimize} flips the sign of scores when \texttt{optimize} is \texttt{min},
    therefore each $y_i = -x_i$ if \texttt{optimize} is \texttt{min}, else $y_i = x_i$.
    This avoids negation of a signed integer.

    Assume scores are non-null, as required by the precondition.
    Then \texttt{max\_sample} on line \ref{fn-max-sample-exact} defines a total ordering on the scores.

    Since the score vector is finite, and \texttt{max\_sample} defines a total ordering,
    then the preconditions for \rustdoc{measurements/report_noisy_top_k/fn}{top} are met.
    Therefore on line \ref{fn-top-exact} \texttt{top} returns the pairs with the top $k$ scores.
    Line \ref{fn-top} then discards the scores, returning only the indices,
    which is the desired output.

    There is one source of error in the function,
    when there are no non-null scores in the input vector.
\end{proof}

\begin{lemma}
    \label{lem:matching-scores}
    The postcondition holds when all scores are the same.
\end{lemma}

\begin{proof}[Proof of Lemma~\ref{lem:matching-scores}]
    By lemma \ref{lem:zero-scale}, the postcondition holds when the scale is zero.

    When all scores are the same, the condition on line \ref{fn-all-same} is met.
    The probability of selecting each subset of $k$ candidates is equal,
    Therefore it is equivalent to randomly select $k$ candidates from the input vector.
    The algorithm returns the top $k$ after shuffling the input vector,
    satisfying the postcondition.
\end{proof}

\begin{proof}[Proof of Theorem~\ref{postcondition}]
    By lemma \ref{lem:zero-scale}, the postcondition holds when the scale is zero,
    and by lemma \ref{lem:matching-scores}, the postcondition holds when all scores are the same.
    We now consider the general case.

    Assume scores are non-null, as required by the precondition.
    Therefore casts on line \ref{fn-cast} should never fail.
    However, if the input data is not in the input domain, and a score is null,
    then line \ref{fn-filter-nan} will filter out failed casts.
    This can be seen as a 1-stable transformation of the input data.

    The algorithm then proceeds to line \ref{fn-normalize-sign}.
    Assuming \texttt{optimize} is \texttt{max}, the line is a no-op, otherwise it negates each score,
    therefore each $y_i = -x_i$ if \texttt{optimize} is \texttt{min}, else $y_i = x_i$.

    The algorithm proceeds to line \ref{fn-init-sample}.
    The output measure has an associated noise distribution that is encoded into the Rust type system via
    \rustdoc{measurements/report\_noisy\_top\_k/trait}{SelectionMeasure}.
    \texttt{MO.random\_variable} on line \ref{fn-rv} creates a random variable \texttt{rv} 
    distributed according to \texttt{MO::RV} and parameterized by \texttt{shift} (the score), and \texttt{scale}.

    To sample from this random variable, line \ref{fn-partial-sample} constructs an instance of 
    \rustdoc{traits/samplers/psrn/struct}{PartialSample}, 
    which represents an infinitely precise sample from the random variable \texttt{rv}.
    We now have an iterator of pairs containing the index and noisy score of each candidate.

    The algorithm proceeds to line \ref{fn-max-sample},
    which defines a reducer based on \rustdoc{traits/samplers/psrn/struct}{PartialSample}\texttt{greater\_than}.
    Assume the scores are non-null, as required by the returned function precondition.
    Then \texttt{max\_sample} on line \ref{fn-max-sample} defines a total ordering on the scores.

    Since the score vector is finite, and \texttt{max\_sample} defines a total ordering,
    then the preconditions for \rustdoc{measurements/report_noisy_top_k/fn}{top} are met.
    Therefore on line \ref{fn-top} \texttt{top} returns the pairs with the top $k$ scores.
    Line \ref{fn-return-indices} then discards the scores, returning only the indices,
    which is the desired output.

    If entropy is exhausted, then the algorithm will return an error from \rustdoc{traits/samplers/psrn/struct}{PartialSample}\texttt{greater\_than}.
    Otherwise, there is one source of error in the function,
    when there are no non-null scores in the input vector.
\end{proof}

The algorithm avoids materializing an infinitely precise sample in memory by comparing finite arbitrary-precision bounds 
on the noisy scores until the lower bound of one noisy score is greater than the upper bound of all others.

\end{document} 

\documentclass{article}
% common styling and macros shared by all proof files

\usepackage[top=1in, right=1in, left=1in, bottom=1.5in]{geometry}

\usepackage{amsmath,amsthm,amsfonts,amssymb,amscd}
\usepackage{listings}
\usepackage{hyperref}
\usepackage{xcolor}
\usepackage{xr}

\usepackage{enumerate} 
\usepackage{physics}
\usepackage{fancyhdr}
\usepackage{hyperref}
\usepackage{graphicx}
\usepackage{tcolorbox}
\usepackage{catchfile}
\usepackage{pdftexcmds}
\usepackage[T1]{fontenc}

% hyperref
\hypersetup{
  colorlinks=true,
  linkcolor=blue,
  linkbordercolor={0 0 1}
}

% \contrib macro to indicate inclusion in "contrib".
\usepackage{tcolorbox}
\newtcolorbox{contrib_box}{colback=red!5!white,colframe=red!75!black}
\newcommand{\contrib}{{\begin{contrib_box}This proof resides in \textbf{``contrib''} because it has not completed the vetting process.\end{contrib_box}}} 

% asOfCommit macro to version a code dependency. Arguments:
%    #1: relative path to file you are dependent on
%    #2: commit hash it was last edited. If outdated, this should be the second hash in the footnoote. Otherwise,
%            git log -n 1 --pretty=format:%h -- path/to/file.rs
\makeatletter
\ifnum\pdf@shellescape=1
   % "private" command that builds a link to a blob
  \newcommand{\linkOpendpBlob}[3]{%
    \href{https://github.com/opendp/opendp/blob/#1/#2#3}{\path{#3} at commit #1}}

  % latex macro expansion has a separate phase for \input evaluation
  %     immediately evaluate a command to write a temp file to ./out containing the current directory
  \immediate\write18{[ ! -f out/cwd.txt ] && (mkdir -p out && git rev-parse --show-prefix | sed "s|_|\@backslashchar\@backslashchar\@backslashchar_|g" > out/cwd.txt)}
  %     ...and then retrieve the current working directory by loading the temp file
  \CatchFileDef\GitWorkingDir{out/cwd.txt}{\endlinechar=-1}

  % command for building the (up to date) or (outdated) status
  \newcommand{\fileStatus}[2]{%
  \setbox0=\hbox{\input|"git --no-pager log -n1 --pretty='\@percentchar H' #1 | grep -E '^#2.*'"\unskip}\ifdim\wd0=0pt
        (outdated\footnote{See new changes with \texttt{git diff #2..\input|"git --no-pager log -n1 --pretty='\@percentchar h' #1" \GitWorkingDir\path{#1}}})\else
        (up to date)\fi
  }

  \newcommand{\asOfCommit}[2]{%
      % permalink the target
      \linkOpendpBlob{#2}{\GitWorkingDir}{#1}
      % conditionally add (outdated) or (up to date) depending on matching commit hash
      \fileStatus{#1}{#2}%
  }
\else
  % simplified command if shell-escape not enabled
  \newcommand{\asOfCommit}[2]{#1 at commit #2 (unknown status\footnote{Shell-escape is not enabled. Enable \texttt{--shell-escape} to check if this proof is up-to-date with the code.})}
\fi
\makeatother

% \vettingPR macro to link a PR. Arguments:
%    #1: PR number
\newcommand{\vettingPR}[1]{\href{https://github.com/opendp/opendp/pull/#1}{Pull Request \##1}}

% for links to rustdoc items in OpenDP. Arguments:
%    #1: path to item, and designation as trait, struct, fn, etc.
%    #2: item name
\makeatletter
\ifnum\pdf@shellescape=1
  % latex macro expansion has a separate phase for \input evaluation
  %     immediately evaluate a command to write a temp file to ./out containing the base path
  \immediate\write18{[ ! -f out/rustdoc.txt ] && mkdir -p out && ([ -z `kpsewhich --var-value OPENDP_RUSTDOC_PORT` ] && echo "https://docs.rs/opendp/`head -n 1 \@backslashchar`git rev-parse --show-toplevel\@backslashchar`/VERSION | sed 's|0.0.0+development|latest|g'`" || echo "http://localhost:`kpsewhich --var-value OPENDP_RUSTDOC_PORT`") > out/rustdoc.txt}
  %     ...and then retrieve the base path by loading the temp file
  \CatchFileDef\OpenDPRustdocBase{out/rustdoc.txt}{\endlinechar=-1}
\else
  % if shell commands are not enabled, just claim latest
  \newcommand{\OpenDPRustdocBase}{https://docs.rs/opendp/latest}
\fi
\makeatother
\newcommand{\rustdoc}[2]{\href{\OpenDPRustdocBase/opendp/#1.#2.html}{\texttt{#2}}}

% for links to external dependencies. Arguments:
%    #1: crate name
%    #2: path to item, and designation as trait, struct, fn, etc.
%    #3: item name
\newcommand{\docsrs}[3]{\href{https://docs.rs/#1/latest/#1/#2.#3.html}{\texttt{#3}}}

% minted (pseudocode)
\definecolor{codegreen}{rgb}{0,0.6,0}
\definecolor{codegray}{rgb}{0.5,0.5,0.5}
\definecolor{codepurple}{rgb}{0.58,0,0.82}
\definecolor{backcolour}{rgb}{0.95,0.95,0.92}

\lstdefinestyle{mystyle}{
    backgroundcolor=\color{backcolour},   
    commentstyle=\color{codegreen},
    keywordstyle=\color{magenta},
    numberstyle=\tiny\color{codegray},
    stringstyle=\color{codepurple},
    basicstyle=\ttfamily\footnotesize,
    breakatwhitespace=false,         
    breaklines=true,                 
    captionpos=b,                    
    keepspaces=true,                 
    numbers=left,                    
    numbersep=5pt,                  
    showspaces=false,                
    showstringspaces=false,
    showtabs=false,                  
    tabsize=2
}

\lstset{style=mystyle}

% common commands
\theoremstyle{definition}
\newtheorem{theorem}{Theorem}[section]
\newtheorem{lemma}[theorem]{Lemma}
\newtheorem{definition}[theorem]{Definition}
\newtheorem{warning}{Warning}
\newtheorem{corollary}{Corollary}
\newtheorem{proposition}{Proposition}
\newtheorem{remark}{Remark}
\newtheorem{observation}{Observation}
\newtheorem{note}{Note}

\newcommand{\vicki}[1]{{ {\color{olive}{(vicki)~#1}}}}
\newcommand{\hanwen}[1]{{ {\color{purple}{(hanwen)~#1}}}}
\newcommand{\zach}[1]{{ {\color{red}{(zach)~#1}}}}

\newcommand{\MultiSet}{\mathrm{MultiSet}}
\newcommand{\len}{\mathrm{len}}
\newcommand{\din}{\texttt{d\_in}}
\newcommand{\dout}{\texttt{d\_out}}
\newcommand{\T}{\texttt{T} }
\newcommand{\F}{\texttt{F} }
\newcommand{\Map}{\texttt{Map}}
\newcommand{\X}{\mathcal{X}}
\newcommand{\Y}{\mathcal{Y}}
\newcommand{\True}{\texttt{True}}
\newcommand{\False}{\texttt{False}}
\newcommand{\clamp}{\texttt{clamp}}
\newcommand{\function}{\texttt{function}}
\newcommand{\float}{\texttt{float }}
\newcommand{\questionc}[1]{\textcolor{red}{\textbf{Question:} #1}}


\newcommand{\validTransformation}[2]{%
  For every setting of the input parameters #1 to #2 such that the given preconditions
  hold, #2 raises an exception (at compile time or run time) or returns a valid transformation. A valid transformation has the following properties:
  \begin{enumerate}
      \item \textup{(Appropriate output domain).} 
      For every element $v$ in \texttt{input\_domain}, $\function(v)$ is in \texttt{output\_domain} or raises a data-independent runtime exception.
      
      \item \textup{(Domain-metric compatibility).} 
      The domain \texttt{input\_domain} matches one of the possible domains listed in the definition of \texttt{input\_metric}, 
      and likewise \texttt{output\_domain} matches one of the possible domains listed in the definition of \texttt{output\_metric}.
      
      \item \textup{(Stability guarantee).} 
      For every pair of elements $u, v$ in \texttt{input\_domain} and for every pair $(\din, \dout)$, 
      where \din\ has the associated type for \texttt{input\_metric} and \dout\ has the associated type for \\ \texttt{output\_metric}, 
      if $u, v$ are \din-close under \texttt{input\_metric} and $\texttt{stability\_map}(\din) \leq \dout$, 
      then $\function(u), \function(v)$ are $\dout$-close under \texttt{output\_metric}.
  \end{enumerate}
}


\newcommand{\validMeasurement}[2]{%
  For every setting of the input parameters #1 to #2 such that the given preconditions
  hold, #2 raises an exception (at compile time or run time) or returns a valid measurement. A valid measurement has the following properties:
  \begin{enumerate}
      \item \textup{(Domain-metric compatibility).} 
      The domain \texttt{input\_domain} matches one of the possible domains listed in the definition of \texttt{input\_metric}.
      
      \item \textup{(Privacy guarantee).} 
      For every pair of elements $u, v$ in \texttt{input\_domain} and for every pair $(\din, \dout)$, 
      where \din\ has the associated type for \texttt{input\_metric} and \dout\ has the associated type for \\ \texttt{output\_measure}, 
      if $u, v$ are \din-close under \texttt{input\_metric} and $\texttt{privacy\_map}(\din) \leq \dout$, 
      then $\function(u), \function(v)$ are $\dout$-close under \texttt{output\_measure}.
  \end{enumerate}
}



\title{\texttt{Bounded Range to zCDP}}
\author{Tudor Cebere}

\begin{document}
\maketitle

\contrib
Proves soundness of \texttt{fn bounded\_range\_to\_zCDP} in \asOfCommit{mod.rs}{0b8f4222}.
This proof is an adaptation of Theorem 5 \href{https://differentialprivacy.org/exponential-mechanism-bounded-range/}{here}, which proves the conversion between bounded range \cite{durfee2019practical} and zCDP.

\subsection{Hoare Triple}
\subsubsection*{Preconditions}
\begin{itemize}
    \item Variable \texttt{eta} must be of type float64
\end{itemize}

\subsubsection*{Pseudocode}

\lstinputlisting[language=Python,firstline=2,escapechar=|]{./pseudocode/br_zcdp.py}

\subsubsection*{Postcondition}
\begin{definition}[Privacy Loss Random Variable]
    Let $M : X^n \to \Omega$ be a randomized mechanism, and let $D, D' \in X^n$ be two neighboring datasets. 
    For each possible output $y \in \Omega$, define the \emph{privacy loss at $y$} as
    \begin{equation}
      f(y) := \log\left(\frac{\Pr[M(D) = y]}{\Pr[M(D') = y]}\right).
    \end{equation}
    Since $M(D)$ is a random outcome in $\Omega$, we can then define the \emph{privacy loss random variable} $Z$ as
    \begin{equation}
      Z := f(M(D)).
    \end{equation}
    In other words, $Z$ is the random variable obtained by first drawing an outcome $y$ from $M(D)$ and then evaluating $f(y)$.
  \end{definition}
  
\begin{definition}[$\eta$-Range Bounded]
\label{def:br}
    Let $M : X^n \to Y$ be a randomized mechanism and let $\eta > 0$. 
    We say that $M$ is $\varepsilon$-range-bounded if $\forall y \in Y$ 
    and any pair of neighboring datasets $D, D' \in X^n$, there exists a $t$ such that:
    \begin{equation}
        Z \in \left[t, t + \eta \right].
    \end{equation}
\end{definition}

\begin{definition}[Concentrated Differential Privacy]
    \label{def:zcdp}
    A randomized algorithm \( M : \mathcal{X}^n \to \mathcal{Y} \) satisfies \(\rho\)-concentrated differential privacy if, for all pairs of inputs \( x, x' \in \mathcal{X}^n \) differing only in the data of a single individual and $\forall \lambda > 0$, we have:
    \begin{equation}
        \mathbb{E}\left[\exp(\lambda Z)\right] \le \exp\left(\lambda(\lambda+1)\rho\right),
    \end{equation}
\end{definition}

\begin{lemma} A privacy loss random variable $Z$ satisfies
    \label{lemma:PRV_lemma_1}
    \begin{equation}
        \mathbb{E}\left[\exp(-Z)\right] = 1
    \end{equation}
    \begin{proof}
    \begin{align}
        \mathbb{E}\left[(-Z)\right] &= \sum_y \mathbb{P}\left[ M(x) = y\right] \exp(-f(y)) \\
        & = \sum_y \mathbb{P}\left[M(x) = y\right] \frac{M(x') = y}{M(x) = y} = 1
    \end{align}
    \end{proof}
\end{lemma}

\begin{lemma} Let M satisfy $\eta$-Bounded Range, then it's associated privacy random variable satisfies:
    \label{lemma:PRV_lemma_2}
    \begin{equation}
        \mathbb{E}\left[Z\right] \leq \frac{1}{8}\eta^2
    \end{equation}

    \begin{proof}
    \begin{align}
        \mathbb{E}\left[\exp(-Z)\right] & \leq \exp(-\mathbb{E}[Z] + \frac{\eta^2}{8}) && (\text{Via Hoeffding, using } Z \in [t, t+\eta]) \\
        \implies \mathbb{E}\left[Z\right] & \leq \frac{\eta^2}{8} - \log{E[\exp(-Z)]} && (\text{rearranging the terms}) \\
        & = \frac{\eta^2}{8} && (\text{using Lemma \ref{lemma:PRV_lemma_1}})
    \end{align}
    \end{proof}
\end{lemma}

\begin{theorem}[Bounded Range implies zCDP]
    If $M$ is $\eta$-bounded range, then it is $\frac{1}{8}\eta^{2}$-concentrated differentially private.
\end{theorem}

\begin{proof}
\begin{align}
    \mathbb{E}\left[\exp(\lambda Z)\right]
    & \leq \exp(\mathbb{E}[Z] \lambda + \frac{\eta^2}{8} \lambda^2) && (\text{Via Hoeffding, using } Z \in [t, t+\eta]) \\
    & \leq \exp(\frac{\eta^2}{8} \lambda + \frac{\eta^2}{8} \lambda^2) && (\text{Via Lemma \ref{lemma:PRV_lemma_2}})
    & \leq \exp(\frac{\eta^2}{8} \lambda + \frac{\eta^2}{8} \lambda^2) \\
    & \leq \exp(\frac{\eta^2}{8} \lambda \left( \lambda + 1\right)) \\
    & \implies \frac{\eta^2}{8}\text{-zCDP} && (\text{Via Definition \ref{def:zcdp}})
\end{align}
\end{proof}

\end{document}
